\documentclass[a4paper,10pt]{article}
\usepackage[utf8]{inputenc}
\usepackage[catalan]{babel}

% a estudiar plis http://www.ling.upenn.edu/advice/latex/qtree/

% Opino que la vida sense color no es vida
\usepackage{color}
\newcommand{\blue}[1]{{\color{blue}#1}}
\newcommand{\red}[1]{{\color{red}#1}}

% Bastant elemental
\usepackage{amssymb, amsmath}

% per crear algoritmes http://minisconlatex.blogspot.com.es/2012/10/pseudocodigo.html
% \usepackage{algpseudocode, program}
\usepackage[lined,boxed,commentsnumbered]{algorithm2e}

% genera arbres
\usepackage{qtree}

\begin{document}
\tableofcontents\newpage
\part{Algoritme}
Orientat a Grafs.
\section{Optimització fins el moment}
\begin{enumerate}
\item No faràs un graf de major profunditat de la demanada
\item Mouràs màxim -1 moviments, més no tenen sentit
\item Només cal moure \red{U} $\uparrow$ ``up'', i \red{R} $\to$ ``right''.
\item Només cal (files $\times$ columnes):
	\begin{itemize}
	\item (1 $\times$ n)
		\subitem \red{U} /1/$\quad$/1 $\to$ (columnes -1)/
	\item (n $\times$ 1)
		\subitem \red{R} /1 $\to$ (files -1)/$\quad$/1/
	\end{itemize}
\item Els fills no repetiran la mateixa direcció a la mateixa posició
\end{enumerate}

\section{Optimització desitjada}
\begin{enumerate}
\item No fer moviments que es contrarestin. No voldria anular tota la feina feta.
\item M'agradaria si la sortida n'és una, fer uns moviments més tancats que la gran majoria.
\end{enumerate}

\section{Planificació}

\subsection{Ordre d'exploració}
\Tree [.1 [.2 [.3 4 5 ] [.6 7 8 ] ] [.9 [.10 11 12 ] [.13 14 15 ] ] ]
Llavors l'algoritme quedarà representat.\\

\subsection{Idea gran}
\begin{algorithm} % http://osl.ugr.es/CTAN/macros/latex/contrib/algorithm2e/doc/algorithm2e.pdf
% defecte python
\SetStartEndCondition{ }{}{}%
\SetKwProg{Fn}{def}{\string:}{}
\SetKwFunction{Range}{range}%%
\SetKw{KwTo}{in}\SetKwFor{For}{for}{\string:}{}%
\SetKwIF{If}{ElseIf}{Else}{if}{:}{elif}{else:}{}%
\SetKwFor{While}{while}{:}{}%{endWhile}%
%\newcommand{\forcond}{$i$ \KwTo\Range{$n$}}
%\AlgoDontDisplayBlockMarkers\SetAlgoNoEnd\SetAlgoNoLine%

% Programes
\SetKwProg{funTot}{Funció}{\string:}{}%{EndFunció}
% Variables
\SetKwData{Matriu}{matriu}
\SetKwData{Canvi}{\red{canvi}}
\SetKwData{mE}{matriuEnd}
\SetKwData{tmp}{temporalPosibleSolucio}
\SetKwData{arr}{array}
\SetKwData{False}{false}
\SetKwData{Profunditat}{level}
% Funcions
\SetKwFunction{MatriuHasNext}{matriu.hasNext}
\SetKwFunction{This}{this}
\SetKwFunction{funRecursiu}{Funció\_Recursiva}
\SetKwFunction{gen}{matriu.MakeGenerator}
\SetKwFunction{mcv}{matriu.canvi}
\SetKwFunction{add}{array.add}
\SetKwFunction{ProfunditatMatriu}{matriu.level}
% In sortida
\SetKwInOut{Input}{input}
\SetKwInOut{Output}{output}
\SetKwInOut{Yield}{yield}

% Programa
\funTot{ \funRecursiu{Canvi a efectuar}}{
\Input{\Canvi, \Matriu}
\Output{Camí per a solucionar el problema}

\BlankLine
\tcc*[h]{Comprovar si el canvi efectuat ja es la solució desitjada}\\
\mcv{\Canvi}\\
\If{\Matriu $\overset{?}{=}$ \mE}{%$ ?^{?^{?^?}}_{\to_{\to_{\to_\to}}} $}{
\arr $= \emptyset$\\
\add{\Canvi}\\
\Yield \arr
\BlankLine
}\tcc{Comprovar si hem d'acabar aquesta branca}
\ElseIf{\ProfunditatMatriu{} $\overset{?}{=}$ \Profunditat}
{
	\Yield \False
}

\BlankLine
\gen{\Canvi}
\BlankLine

	\While{\MatriuHasNext{}}
	{
		\BlankLine
		\arr = \funRecursiu{\Canvi}
		\BlankLine
		\tcc{Comprovar si és solució}
		\If{\arr}
		{
			\add{\Canvi}\\
			\Yield \arr
		}
	}

	\BlankLine
	\tcc{Si arribem aquí, malament.}
	\Yield \False
}
\end{algorithm}

\subsection{Generar el iterador}

\part{Comunicació}
\section{Entrada d'informació}
\begin{itemize}
\item 1era línia
	\subitem Màxim moviments permesos
\item Matriu entrada
\item Espai en blanc
\item Matriu sortida
\end{itemize}
Separats per espais, contenint lletres, en referència a la inicial del color.

\section{Sortida d'informació}
encarà no hi he pensat

\part{Warnings}
\section{Generar noves matrius}
Quan faig la crida recursiva, preferiria controlar amb gran nivell que està passant, vull evitar el gran problema d'editar la mateixa
matriu

\part{Millores}
Canviar l'algoritme per aquest:
\Tree [.1 [.2 [.4 8 9 ] [.5 10 11 ] ] [.3 [.6 12 13 ] [.7 14 15 ] ] ]
D'aquesta forma no caldria indicar la profunditat màxima
\end{document}
