\documentclass[a4paper,10pt]{article}
\usepackage[utf8]{inputenc}
\usepackage[catalan]{babel}

% a estudiar plis http://www.ling.upenn.edu/advice/latex/qtree/

% Opino que la vida sense color no es vida
\usepackage{color}
\newcommand{\blue}[1]{{\color{blue}#1}}
\newcommand{\red}[1]{{\color{red}#1}}

% Bastant elemental
\usepackage{amssymb, amsmath}

% per crear algoritmes http://minisconlatex.blogspot.com.es/2012/10/pseudocodigo.html
% \usepackage{algpseudocode, program}
\usepackage[lined,boxed,commentsnumbered]{algorithm2e}

% genera arbres
\usepackage{qtree}

\begin{document}
\tableofcontents\newpage
\part{Optimitzacions}
\begin{enumerate}
\item Només cal moure \red{$U$} $\uparrow$ ``up'', i \red{$R$} $\to$ ``right''.
% és subtil, però és bo tenir-lo present
\item Els fills no repetiran la mateixa direcció
\item Generador
	\begin{itemize}
	\item (1 $\times$ n)
		\subitem \red{U}$\quad$ 1 $\to$ (columnes -1)
	\item (m $\times$ 1)
		\subitem \red{R}$\quad$ 1 $\to$ (files -1)
	\end{itemize}
\end{enumerate}

\section{Planificació}
Ja que suposem que no coneixem la profunditat màxima.
\Tree [.1 [.2 [.4 8 9 ] [.5 10 11 ] ] [.3 [.6 12 13 ] [.7 14 15 ] ] ]

\section{Algoritme}
\subsection{Estil nodes, grafs, arbre}
\paragraph{Idea bàsica}
Generarem un objecte el qual cridarà als seus fills, i esperarà a la solució desitjada.
\paragraph{Problemes aparents}
Tindrà tot els pares i fills en memòria, quan realment només són necessaris els seus fills.\\
Però no estic realment segur del problema, ja que d'altra forma no sé com podria guardar els seus pares.\\

Per exemple, jo només vull com a sortida 0R2, 3U4, etc, això gairebé no ocupa espai.
\paragraph{Avantatges}
Sembla més fàcil d'entendre.
\newpage
\subsubsection{Mostra de l'algoritme}
\begin{algorithm} % http://osl.ugr.es/CTAN/macros/latex/contrib/algorithm2e/doc/algorithm2e.pdf
% defecte python
\SetStartEndCondition{ }{}{}%
\SetKwProg{Fn}{def}{\string:}{}
\SetKwFunction{Range}{range}%%
\SetKw{KwTo}{in}\SetKwFor{For}{for}{\string:}{}%
\SetKwIF{If}{ElseIf}{Else}{if}{:}{elif}{else:}{}%
\SetKwFor{While}{while}{:}{}%{endWhile}%
% Això és un copiar i pegar del anterior tex que vaig fer.

% Definir tot el que necessito
% Programes
\SetKwProg{funTot}{Funció}{\string:}{}%{EndFunció}
% Variables
\SetKwData{Directori}{directori}
\SetKwData{MatriuEntrada}{matriuEntrada}
\SetKwData{MatriuSolucio}{matriuSolució}
\SetKwData{Solucio}{solució}
\SetKwData{TamanyNombreFills}{n}
\SetKwData{VarGenerador}{\red{generador}}
\SetKwData{Array}{arr}
\SetKwData{Auxiliar}{aux}
\SetKwData{Int}{i}
\SetKwData{Anar}{\red{c\blue{am}í}}
% Funcions
\SetKwFunction{funBegin}{Funció\_Recursiva}
\SetKwFunction{funNext}{\blue{Funció\_Recorre\_Recorregut}}
\SetKwFunction{Llegeix}{Llegeix}
\SetKwFunction{Generador}{Generador}
\SetKwFunction{Next}{next}
\SetKwFunction{Xrange}{xrange}
\SetKwFunction{Append}{append}
% In sortida
\SetKwInOut{Input}{input}
\SetKwInOut{Output}{output}
\SetKwInOut{Yield}{yield}
\SetKwInOut{Return}{return}

% Comença tota la màgia
% Programa
\funTot{ \funBegin{\Directori} }
{
	\Input{\MatriuEntrada, \MatriuSolucio}
	\Output{\Solucio}
	\MatriuEntrada, \MatriuSolucio = \Llegeix{\Directori }\\
	\VarGenerador = \Generador{\MatriuEntrada, \red{null}}\\
	\TamanyNombreFills = \VarGenerador.\Next{}\\
	\Array = []\\
\tcc*[h]{Totes les variables necessàries inicialitzades}\\
% fins aquí per arrancar el programa en si

	\BlankLine
\tcc*[h]{Comencem a inicialitzar els fills}\\
	\For{\Int $\in$ \Xrange{\TamanyNombreFills} }
	{
		\Array.\Append{\funNext{\VarGenerador.\Next{}}}
	}
	\BlankLine

\tcc*[h]{El programa en si, aparentment infinit}\\
	\While{ 1 }
	{
		\For{\Int $\in$ \Xrange{\TamanyNombreFills} }
		{
			\Auxiliar = \Array[\Int].\Next{}\\
			\If{\Auxiliar}
			{
				\Return{\Auxiliar}
			}
		}
	}
}
\BlankLine
\BlankLine
\funTot{ \funNext{\Anar} }
{
\tcc*[h]{Per quan ha acabat el programa}\\
	\If{\MatriuEntrada.\Next{\Anar} $\overset{?}{=}$ \MatriuSolucio}
	{
		\Yield{$[$\Anar$]$}
	}
\BlankLine
\tcc*[h]{Inicialitzem valors}\\
	\VarGenerador = \Generador{\MatriuEntrada, \Anar}\\
	\TamanyNombreFills = \VarGenerador.\Next{}\\
	\Array = []
\BlankLine
\tcc*[h]{Generar els fills}\\
	\For{\Int $\in$ \Xrange{\TamanyNombreFills}}
	{
		\Array.\Append{\funNext{\VarGenerador.\Next{}}}
	}
\BlankLine
\tcc*[h]{El programa en si}\\
	\While{ 1 }
	{
		\Yield{\red{False, False}}
		\BlankLine
		\For{\Int $\in$ \Xrange{\TamanyNombreFills} }
		{
			\Auxiliar, \Anar = \Array[\Int].\Next{}\\
			\If{\Auxiliar}
			{
				\Return{\blue{True}, \Auxiliar + $[$\Anar$]$}
			}
		}
	}
}
\end{algorithm}

% El que m'ha dit en Martin, i realment crec que hauria de ser així el meu programa. En sap molt aquest noi.
\subsection{Estil cua, grafs}
\paragraph{Idea bàsica}
Tens una cua, treus un element, llavors obtens tots els seus fills, i si veus que no son la solució desitjada els tornes dins la cua.
\paragraph{Problemes aparents}
Sembla més difícil d'implementar perquè no he madurat cap idea per a fer-ho per aquest camí.
\paragraph{Avantatges}
L'execució del programa serà considerablement molt més ràpida.\\

Sembla que podríem aconseguir que sortiríem tant del inici com del final, i anar comprovant si coincideixen els resultats, si és el cas has
acabat.\\
Aparentment donaria com a resultat que no seria necessari aprofundir tant els arbres.\\
Pega, que hauries de trobar un bon sistema de comparació per no fer-ho excessivament lent en si el programa.

\end{document}
